\usepackage{amsmath,amstext,amssymb,amsfonts}
\usepackage{color,graphicx}
\usepackage{tabularx}
\usepackage{verbatim}

% theorem environments
\usepackage{amsthm}
\usepackage{ifdraft}


%\usepackage[color]{showkeys}

\usepackage{nicefrac}

% write fractions using \frac, \tfrac, \nfrac, or \ffrac (avoid ffrac)
\newcommand{\flatfrac}[2]{#1/#2}
\newcommand{\ffrac}{\flatfrac}
\newcommand{\nfrac}{\nicefrac}

\usepackage{microtype}

\newtheorem{theorem}{Theorem}[section]

\newtheorem{Claim}[theorem]{Claim}
\newtheorem{subclaim}{Claim}[theorem]
\newtheorem{proposition}[theorem]{Proposition}
\newtheorem{lemma}[theorem]{Lemma}
\newtheorem{corollary}[theorem]{Corollary}
\newtheorem{conjecture}[theorem]{Conjecture}
\newtheorem{observation}[theorem]{Observation}
\newtheorem{fact}[theorem]{Fact}
\newtheorem{hypothesis}[theorem]{Hypothesis}
\newtheorem*{hypothesis*}{Hypothesis}

\theoremstyle{definition}
\newtheorem{definition}[theorem]{Definition}
\newtheorem{construction}[theorem]{Construction}
\newtheorem{example}[theorem]{Example}
\newtheorem{algorithm}[theorem]{Algorithm}
\newtheorem{SDP}[theorem]{SDP}
\newtheorem{problem}[theorem]{Problem}
\newtheorem{protocol}[theorem]{Protocol}
\newtheorem{remark}[theorem]{Remark}
\newtheorem{assumption}[theorem]{Assumption}



\usepackage{prettyref}
\newcommand{\savehyperref}[2]{\texorpdfstring{\hyperref[#1]{#2}}{#2}}

\newrefformat{eq}{\savehyperref{#1}{\textup{(\ref*{#1})}}}
\newrefformat{lem}{\savehyperref{#1}{Lemma~\ref*{#1}}}
\newrefformat{def}{\savehyperref{#1}{Definition~\ref*{#1}}}
\newrefformat{thm}{\savehyperref{#1}{Theorem~\ref*{#1}}}
\newrefformat{cor}{\savehyperref{#1}{Corollary~\ref*{#1}}}
\newrefformat{cha}{\savehyperref{#1}{Chapter~\ref*{#1}}}
\newrefformat{sec}{\savehyperref{#1}{Section~\ref*{#1}}}
\newrefformat{app}{\savehyperref{#1}{Appendix~\ref*{#1}}}
\newrefformat{tab}{\savehyperref{#1}{Table~\ref*{#1}}}
\newrefformat{fig}{\savehyperref{#1}{Figure~\ref*{#1}}}
\newrefformat{hyp}{\savehyperref{#1}{Hypothesis~\ref*{#1}}}
\newrefformat{alg}{\savehyperref{#1}{Algorithm~\ref*{#1}}}
\newrefformat{sdp}{\savehyperref{#1}{SDP~\ref*{#1}}}
\newrefformat{rem}{\savehyperref{#1}{Remark~\ref*{#1}}}
\newrefformat{item}{\savehyperref{#1}{Item~\ref*{#1}}}
\newrefformat{step}{\savehyperref{#1}{step~\ref*{#1}}}
\newrefformat{conj}{\savehyperref{#1}{Conjecture~\ref*{#1}}}
\newrefformat{fact}{\savehyperref{#1}{Fact~\ref*{#1}}}
\newrefformat{prop}{\savehyperref{#1}{Proposition~\ref*{#1}}}
\newrefformat{claim}{\savehyperref{#1}{Claim~\ref*{#1}}}
\newrefformat{relax}{\savehyperref{#1}{Relaxation~\ref*{#1}}}
\newrefformat{red}{\savehyperref{#1}{Reduction~\ref*{#1}}}
\newrefformat{part}{\savehyperref{#1}{Part~\ref*{#1}}}
\newrefformat{prob}{\savehyperref{#1}{Problem~\ref*{#1}}}
\newrefformat{ass}{\savehyperref{#1}{Assumption~\ref*{#1}}}



% {{{ sref }}}

% short section reference
\newcommand{\Sref}[1]{\hyperref[#1]{\S\ref*{#1}}}


% nicer times font - saves pages
\usepackage[varg]{txfonts}
%\usepackage{times}

% nice mathbb fonts
\renewcommand{\mathbb}{\varmathbb}

% switch to fullpage for final version
% \ifoptionfinal{
 %  \usepackage{fullpage}
 %}

% slanted inequality signs
\renewcommand{\leq}{\leqslant}
\renewcommand{\le}{\leqslant}
\renewcommand{\geq}{\geqslant}
\renewcommand{\ge}{\geqslant}

% shows keys of labels, references, and citations
%\usepackage[color]{showkeys}

% bold math package - provides command to use boldmath font
\usepackage{bm}

% to define text macros
\usepackage{xspace}

% load hyperref
\usepackage[pdftex,pagebackref,colorlinks,linkcolor=blue,filecolor = blue, citecolor = blue, urlcolor  = blue]{hyperref}

% \usepackage[top=2cm, bottom=2cm, left=1.6cm, right=1.6cm]{geometry}


% {{{ boxedminipage }}}
\usepackage{boxedminipage}

\newenvironment{mybox}
{\center \noindent\begin{boxedminipage}{1.0\linewidth}}
{\end{boxedminipage}
\noindent
}


% punctation at the end of displayed formulas
\newcommand{\mper}{\,.}
\newcommand{\mcom}{\,,}


% boldface vectors
\renewcommand{\vec}[1]{{\bm{#1}}}

% prime/tilde vector
\newcommand{\pvec}[1]{\vec{#1}'}
\newcommand{\ppvec}[1]{\vec{#1}''}
\newcommand{\tvec}[1]{{\tilde{\vec{#1}}}}

% parentheses
\newcommand{\paren}[1]{\left(#1 \right )}
\newcommand{\Paren}[1]{\left(#1 \right )}

% brackets
\newcommand{\brac}[1]{[#1 ]}
\newcommand{\Brac}[1]{\left[#1\right]}

% set braces
\newcommand{\set}[1]{\left\{#1\right\}}
\newcommand{\Set}[1]{\left\{#1\right\}}


% absolute value sign
\newcommand{\abs}[1]{\left\lvert#1\right\rvert}
\newcommand{\Abs}[1]{\left\lvert#1\right\rvert}

% ceil floor
\newcommand{\ceil}[1]{\lceil #1 \rceil}
\newcommand{\floor}[1]{\lfloor #1 \rfloor}

% norm
\newcommand{\norm}[1]{\left\lVert#1\right\rVert}
\newcommand{\Norm}[1]{\left\lVert#1\right\rVert}
\newcommand{\fnorm}[1]{\norm{#1}_F}


% define symbol for definition
\newcommand{\defeq}{\stackrel{\textup{def}}{=}}
\newcommand{\iseq}{\stackrel{\textup{?}}{=}}
\newcommand{\isgeq}{\stackrel{\textup{?}}{\geq}}
\newcommand{\isleq}{\stackrel{\textup{?}}{\leq}}
% big vertical space
\newcommand{\vbig}{\vphantom{\bigoplus}}

% linear algebra
\newcommand{\inprod}[1]{\left\langle #1\right\rangle}

% norm
\newcommand{\snorm}[1]{\norm{#1}^2}

% L2 norm
\newcommand{\normt}[1]{\norm{#1}_{\scriptstyle 2}}
\newcommand{\snormt}[1]{\norm{#1}^2_2}

% Projection Operators
\newcommand{\projsymb}{\Pi}
\newcommand{\proj}[2]{\projsymb_{#1}\paren{#2}}
\newcommand{\projperp}[2]{\projsymb_{#1}^{\perp}\paren{#2}}
\newcommand{\projpar}[2]{\projsymb_{#1}^{\parallel}\paren{#2}}

% norms
\newcommand{\normo}[1]{\norm{#1}_{\scriptstyle 1}}
\newcommand{\normi}[1]{\norm{#1}_{\scriptstyle \infty}}
\newcommand{\normb}[1]{\norm{#1}_{\scriptstyle \square}}

% number sets
\newcommand{\Z}{{\mathbb Z}}
\newcommand{\N}{{\mathbb Z}_{\geq 0}}
\newcommand{\R}{\mathbb R}
\newcommand{\Rnn}{\R_+}


% common operators
\newcommand{\Inf}{\ensuremath{\sf{Inf}}}
\newcommand{\sdp}{{\sf SDP }}
\newcommand{\las}{\ensuremath{\sf{Lasserre}}}
\newcommand{\qp}{\mathrm{qp}}
\newcommand{\opt}{{\sf OPT}}
\newcommand{\OPT}{{\sf OPT}}
\newcommand{\lp}{{\sf LP}}
\newcommand{\LP}{{\sf LP}}
\newcommand{\val}{{\sf VAL}}
\newcommand{\cost}{{\sf cost}}
\newcommand{\csp}{{\sf CSP}}
\newcommand{\sse}{{\sf SSE}}

\newcommand{\seteq}{\mathrel{\mathop:}=}
\newcommand{\subjectto}{\text{subject to}}
\newcommand{\card}{\abs}
\newcommand{\Card}{\Abs}
\newcommand{\half}{\nfrac{1}{2}}





% probability symbols
\newcommand{\Esymb}{\mathbb{E}}
\newcommand{\Psymb}{\mathbb{P}}
\newcommand{\Vsymb}{\mathbb{V}}
\newcommand{\Isymb}{\mathbb{I}}
\DeclareMathOperator*{\E}{\Esymb}
\DeclareMathOperator*{\Var}{{\sf Var}}
\DeclareMathOperator*{\ProbOp}{\Psymb}
\newcommand{\var}[1]{\Var \left[#1\right]}

%conditioning 
\newcommand{\given}{\mathrel{}\middle|\mathrel{}}
\newcommand{\Given}{\given}

% probability of an event \prob{e}=IP{e}
\newcommand{\prob}[1]{\ProbOp\Brac{#1}}
\newcommand{\Prob}[1]{\ProbOp\Brac{#1}}

% expectation of variable \ex{X} = IE[X]
\newcommand{\ex}[1]{\E\brac{#1}}
\newcommand{\Ex}[1]{\E\Brac{#1}}

\renewcommand{\Pr}[1]{\ProbOp\Brac{#1}}
\newcommand{\pr}[2]{\ProbOp_{#1}\Brac{#2}}


\newcommand{\ind}[2]{\Isymb_{#1}\brac{#2}}
\newcommand{\Ind}[1]{\Isymb\Brac{#1}}

\newcommand{\varex}[1]{\E\paren{#1}}
\newcommand{\varEx}[1]{\E\Paren{#1}}
\newcommand{\eset}{\emptyset}
\newcommand{\e}{\epsilon}

% superscript with parentheses
\newcommand{\super}[2]{#1^{\paren{#2}}}

% bits
\newcommand{\bits}{\{0,1\}}


% author notes macros
\definecolor{DSgray}{cmyk}{0,0,0,0.7}
\newcommand{\Authornote}[2]{{\small\textcolor{red}{\sf$<<<${  #1: #2 }$>>>$}}}
\newcommand{\Authormarginnote}[2]{\marginpar{\parbox{2cm}{\raggedright\tiny \textcolor{DSgray}{#1: #2}}}}

% disable author notes when final option is present
%\ifoptionfinal{
 % \renewcommand{\Authornote}[2]{}
 % \renewcommand{\Authormarginnote}[2]{}


\newcommand{\Anote}{\Authornote{Anand}}
%\newcommand{\Amarginnote}{\Authormarginnote{Anand}}
%}

% more macros

\let\e\varepsilon

%%%%%%%%%%%%%% Problems 
\newcommand{\problemmacro}[1]{\texorpdfstring{\textsc{#1}}{#1}\xspace}

\newcommand{\mla}{\problemmacro{Minimum Linear Arrangement}}
\newcommand{\BalancedSeparator}{\problemmacro{Balanced Separator}}
\newcommand{\UGCexpand}{\problemmacro{Expanding Unique Games}}
\newcommand{\sparsestcut}{\problemmacro{Sparsest Cut}}
\newcommand{\smallsetexpansion}{\problemmacro{Small Set Expansion}}
\newcommand{\uniquegames}{\problemmacro{Unique Games}}




% {{{ alphabet }}}

\newcommand{\cA}{\mathcal A}
\newcommand{\cB}{\mathcal B}
\newcommand{\cC}{\mathcal C}
\newcommand{\cD}{\mathcal D}
\newcommand{\cE}{\mathcal E}
\newcommand{\cF}{\mathcal F}
\newcommand{\cG}{\mathcal G} 
\newcommand{\cH}{\mathcal H} 
\newcommand{\cI}{\mathcal I} 
\newcommand{\cJ}{\mathcal J} 
\newcommand{\cK}{\mathcal K}
\newcommand{\cL}{\mathcal L}
\newcommand{\cM}{\mathcal M}
\newcommand{\cN}{\mathcal N}
\newcommand{\cO}{\mathcal O}
\newcommand{\cP}{\mathcal P}
\newcommand{\cQ}{\mathcal Q}
\newcommand{\cR}{\mathcal R}
\newcommand{\cS}{\mathcal S}
\newcommand{\cT}{\mathcal T}
\newcommand{\cU}{\mathcal U}
\newcommand{\cV}{\mathcal V}
\newcommand{\cW}{\mathcal W}
\newcommand{\cX}{\mathcal X}
\newcommand{\cY}{\mathcal Y}
\newcommand{\cZ}{\mathcal Z}

\newcommand{\bbB}{\mathbb B}
\newcommand{\bbS}{\mathbb S}
\newcommand{\bbR}{\mathbb R}
\newcommand{\bbZ}{\mathbb Z}
\newcommand{\bbI}{\mathbb I}
\newcommand{\bbQ}{\mathbb Q}
\newcommand{\bbP}{\mathbb P}
\newcommand{\bbE}{\mathbb E}

\newcommand{\sfE}{\mathsf E}


% {{{ names }}}
% Hungarian/Polish/East European names 
\newcommand{\Erdos}{Erd\H{o}s\xspace}
\newcommand{\Renyi}{R\'enyi\xspace}
\newcommand{\Lovasz}{Lov\'asz\xspace}
\newcommand{\Juhasz}{Juh\'asz\xspace}
\newcommand{\Bollobas}{Bollob\'as\xspace}
\newcommand{\Furedi}{F\"uredi\xspace}
\newcommand{\Komlos}{Koml\'os\xspace}
\newcommand{\Luczak}{\L uczak\xspace}
\newcommand{\Kucera}{Ku\v{c}era\xspace}
\newcommand{\Szemeredi}{Szemer\'edi\xspace}
\newcommand{\Hastad}{H{\aa}stad\xspace}







\newcommand{\etal}{et. al.}
\newcommand{\bigO}{\mathcal{O}}
\newcommand{\bigo}[1]{\bigO\left(#1\right)}
\newcommand{\tbigO}{\tilde{\mathcal{O}}}
\newcommand{\tbigo}[1]{\tbigO\left(#1\right)}
\newcommand{\tensor}{\otimes}


\newcommand{\argmax}{{\sf argmax}}
\newcommand{\argmin}{{\sf argmin}}
\newcommand{\poly}{{\sf poly}}
\newcommand{\polylog}{{\sf polylog}}
\newcommand{\supp}{{\sf supp}}

\newcommand{\U}{\bar{u}}
\newcommand{\V}{\bar{v}}
\newcommand{\W}{\bar{w}}

\newcommand{\rank}{{\sf rank}}
\newcommand{\tk}{t_{1/k}} % gausssian cap

\newcommand{\yes}{\textsc{Yes}\xspace}
\newcommand{\no}{\textsc{No}\xspace}


